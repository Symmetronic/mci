% Definiere Dokumenttyp
\documentclass[12pt,a4paper]{article}

\usepackage[utf8]{inputenc}
\usepackage[ngerman]{babel}

% Caption Abstand angepasst
\usepackage{caption}
\captionsetup[table]{aboveskip=4pt}
\captionsetup[table]{belowskip=4pt}

% Bilder
\usepackage{graphicx}
\usepackage{float}
\usepackage{subfigure}

% Hurenkinder und Schusterjungen verhindern
\clubpenalty10000
\widowpenalty10000
\displaywidowpenalty=10000

% Titel und Autor
\newcommand{\theauthor}{Maxime Fleury und Simon Hochholzer}
\newcommand{\thetitle}{Prototyping und TLX}

% PDF-Eigenschaften
\usepackage{hyperref}
\hypersetup{
  bookmarks = true, % zeige Lesezeichenleiste
  bookmarksnumbered = false,
  bookmarksopen = false,
  breaklinks = true,
  backref = true,
  pdftoolbar = true, % zeige Acrobat's Toolbar
  pdfmenubar = true, % zeige Acrobat's Menü
  pdffitwindow = false, % Fenster an Seite anpassen?
  pdfstartview = {FitH}, % Seitenbreite-Ansicht
  pdftitle = {\thetitle}, % Titel
  pdfauthor = {\theauthor}, % Autor
  pdfcreator = {Atom}, % Dokument-Ersteller
  pdfproducer = {LaTeX}, % Dokument-Produzent
  pdfnewwindow = true, % Links in neuem Fenster
  pdfborder = {0 0 0},
  colorlinks = true, % false: boxed links; true: colored links
  linkcolor = black, % Farbe von internen Links
  citecolor = black, % Farbe von Literatur-Links
  filecolor = black, % Farbe von Datei-Links
  urlcolor = black % Farbe von externen Links
}

% Definiere Autoren und Titel
\author{\theauthor}
\title{\thetitle}

% Das eigentliche Dokument
\begin{document}
    \maketitle

    \section{Zusammenfassung der Daten}

Zur Evaluation des Prototypen wurde mit drei Probanden das Thinking Aloud Verfahren in Kombination mit dem TLX-Test verwendet. Die Daten der TLX-Tests der einzelnen Probanden sind in den Tabellen \ref{tab:tlx-proband1}, \ref{tab:tlx-proband2} und \ref{tab:tlx-proband3} aufgelistet.

% TLX-Daten Proband 1
\begin{table}[htb]
    \caption{\label{tab:tlx-proband1}Daten des TLX-Tests für Proband 1.}
    \centering
    \begin{tabular}{l r r}
        Proband 1 & Bewertung & Wichtung \\
        \hline
        Geistige Anforderung & 20 & 0.06666667 \\
        Körperliche Anforderung & 5 & 0 \\
        Zeitliche Anforderung & 85 & 0.2 \\
        Leistung & 40 & 0.26666667 \\
        Anstrengung & 20 & 0.13333333 \\
        Frustration & 85 & 0.33333333 \\
        \hline
        Gesamtbeanspruchung & 60.00 & \\
    \end{tabular}
\end{table}

% TLX-Daten Proband 2
\begin{table}[htb]
    \caption{\label{tab:tlx-proband2}Daten des TLX-Tests für Proband 2.}
    \centering
    \begin{tabular}{l r r}
        Proband 1 & Bewertung & Wichtung \\
        \hline
        Geistige Anforderung & 35 & 0.2 \\
        Körperliche Anforderung & 10 & 0.13333333 \\
        Zeitliche Anforderung & 30 & 0.06666667 \\
        Leistung & 65 & 0.13333333 \\
        Anstrengung & 20 & 0.2 \\
        Frustration & 55 & 0.26666666 \\
        \hline
        Gesamtbeanspruchung & 37.67 & \\
    \end{tabular}
\end{table}

% TLX-Daten Proband 3
\begin{table}[htb]
    \caption{\label{tab:tlx-proband3}Daten des TLX-Tests für Proband 3.}
    \centering
    \begin{tabular}{l r r}
        Proband 1 & Bewertung & Wichtung \\
        \hline
        Geistige Anforderung & 25 & 0.06666667 \\
        Körperliche Anforderung & 5 & 0.2 \\
        Zeitliche Anforderung & 25 & 0.2 \\
        Leistung & 30 & 0.06666667 \\
        Anstrengung & 35 & 0.13333333 \\
        Frustration & 35 & 0.33333333 \\
        \hline
        Gesamtbeanspruchung & 26.00 & \\
    \end{tabular}
\end{table}

Die absoluten (Abbildung \ref{fig:absolut}) und relativen, auf 1 normierten, (Abbildung \ref{fig:relativ}) Werte aus den TLX-Tests der Probanden wurden gegenübergestellt.

% Absolute Beanspruchungen
\begin{figure}[htb]
    \centering
    \subfigure[\label{fig:absolut}Gegenüberstellung der absoluten Beanspruchungen der Probanden aus den TLX-Tests.]{
        \includegraphics[width=\textwidth]{../diagramme/absolute_beanspruchung.png}
    }

    \subfigure[\label{fig:relativ}Gegenüberstellung der relativen Beanspruchungen der Probanden aus den TLX-Tests durch Normierung der Gesamtbeanspruchung auf 1.]{
        \includegraphics[width=\textwidth]{../diagramme/relative_beanspruchung.png}
    }
\end{figure}


    \section{Interpretation}

Betrachtet man die absolute Beanspruchung, ist eindeutig zu erkennen, dass die Beanspruchung zwischen den Probanden stark variiert. Dies bezieht sich auf praktisch alle Werte. Dabei nahmen die Probanden die verschiedenen Arten der Beanspruchung auf teils komplett unterschiedliche Art und Weise wahr. Proband 1 empfand die beispielsweise die zeitliche Beanspruchung als auch den Frustrationsfaktor wesentlich höher als Proband 2 und 3, dies trifft auch auf die Gesamtbeanspruchung zu. Allerdings lag die geistige Beanspruchung als auch die Anstrengung bei Proband 1 niedriger als bei den anderen beiden Probanden. Allgemein sind sich die Werte von Proband 2 und 3 recht ähnlich.
Interessanter ist hier jedoch die relative Beanspruchung, sie stellt besser dar, in welchem Verhältnis die verschiedenen Anforderungen zueinanderstehen. Teils unterscheiden sich die Werte zwischen den einzelnen Probanden stark voneinander, insgesamt wurden die Anstrengung und vor allem körperliche Anforderung aber als sehr niedrig angesehen. Die Leistung und zeitliche Anforderung wurden als stärkere Belastung bezeichnet, wobei jedoch bei beiden Werten jeweils ein Proband deutlich niedrigere Werte als die anderen beiden aufwies. Am deutlichsten ist jedoch der extrem hohe Frustfaktor zu erkennen, der bei allen drei Probanden als der mit Abstand gravierendste Faktor angesehen wurde. Hieraus ist sehr deutlich zu erkennen, dass die Probanden weniger mit der Zeit, körperlicher oder geistiger Anforderung Probleme aufwiesen, am meisten frustrierte sie die Bedienung des Rekorders. Hieraus lässt sich sehr deutlich ableiten, dass die Bedienung des Rekorders insgesamt als negativ empfunden wurde. Dies kann sich auf mehrere Ursachen zurückführen lassen, die jedoch nicht eindeutig geklärt sind. Mögliche Ursachen wären ein zu komplizierter oder verwirrender Umgang mit dem Recorder.


    \section{Konsequenzen}

Der von den Probanden als sehr hoch wahrgenommene Frustfaktor zeigt eindeutig, dass die Bedienung des Recorders in der Praxis dringend überarbeitet werden müsste. Vor allem die Vorgehensweise, dass bei mehreren Aktionen derselbe Knopf mehrmals betätigt werden musste, führte bei den Probanden zu Verwirrung und Unverständnis. Da es sich um einen digitalen Prototypen handelte, wurden einige Probanden weiterhin dadurch negativ überrascht, dass für das Einlegen eines Tapes ein weiterer Knopfdruck erforderlich ist. Auch die doppelte Funktion des „Stop/Eject“-Buttons wurde als negativ empfunden.
Eine Vereinfachung in Form von weniger nötigen Aktionen könnte den Frustfaktor jedoch absenken und somit die Bedienung des Recorders wesentlich angenehmer gestalten.

\end{document}
