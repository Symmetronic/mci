\section{Interpretation}

Betrachtet man die absolute Beanspruchung, ist eindeutig zu erkennen, dass die Beanspruchung zwischen den Probanden stark variiert. Dies bezieht sich auf praktisch alle Werte. Dabei nahmen die Probanden die verschiedenen Arten der Beanspruchung auf teils komplett unterschiedliche Art und Weise wahr. Proband 1 empfand die beispielsweise die zeitliche Beanspruchung als auch den Frustrationsfaktor wesentlich höher als Proband 2 und 3, dies trifft auch auf die Gesamtbeanspruchung zu. Allerdings lag die geistige Beanspruchung als auch die Anstrengung bei Proband 1 niedriger als bei den anderen beiden Probanden. Allgemein sind sich die Werte von Proband 2 und 3 recht ähnlich.
Interessanter ist hier jedoch die relative Beanspruchung, sie stellt besser dar, in welchem Verhältnis die verschiedenen Anforderungen zueinanderstehen. Teils unterscheiden sich die Werte zwischen den einzelnen Probanden stark voneinander, insgesamt wurden die Anstrengung und vor allem körperliche Anforderung aber als sehr niedrig angesehen. Die Leistung und zeitliche Anforderung wurden als stärkere Belastung bezeichnet, wobei jedoch bei beiden Werten jeweils ein Proband deutlich niedrigere Werte als die anderen beiden aufwies. Am deutlichsten ist jedoch der extrem hohe Frustfaktor zu erkennen, der bei allen drei Probanden als der mit Abstand gravierendste Faktor angesehen wurde. Hieraus ist sehr deutlich zu erkennen, dass die Probanden weniger mit der Zeit, körperlicher oder geistiger Anforderung Probleme aufwiesen, am meisten frustrierte sie die Bedienung des Rekorders. Hieraus lässt sich sehr deutlich ableiten, dass die Bedienung des Rekorders insgesamt als negativ empfunden wurde. Dies kann sich auf mehrere Ursachen zurückführen lassen, die jedoch nicht eindeutig geklärt sind. Mögliche Ursachen wären ein zu komplizierter oder verwirrender Umgang mit dem Recorder.
