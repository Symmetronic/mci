\section{Konsequenzen}

Der von den Probanden als sehr hoch wahrgenommene Frustfaktor zeigt eindeutig, dass die Bedienung des Recorders in der Praxis dringend überarbeitet werden müsste. Vor allem die Vorgehensweise, dass bei mehreren Aktionen derselbe Knopf mehrmals betätigt werden musste, führte bei den Probanden zu Verwirrung und Unverständnis. Da es sich um einen digitalen Prototypen handelte, wurden einige Probanden weiterhin dadurch negativ überrascht, dass für das Einlegen eines Tapes ein weiterer Knopfdruck erforderlich ist. Auch die doppelte Funktion des „Stop/Eject“-Buttons wurde als negativ empfunden.
Eine Vereinfachung in Form von weniger nötigen Aktionen könnte den Frustfaktor jedoch absenken und somit die Bedienung des Recorders wesentlich angenehmer gestalten.